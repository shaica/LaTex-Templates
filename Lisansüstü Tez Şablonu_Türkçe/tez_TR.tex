\documentclass[12pt]{report}

%--------------------------------------------------------------------------------------------------------------------------------------------------
%----------------------------------------------------------------Preamble-----------------------------------------------------------------------
%--------------------------------------------------------------------------------------------------------------------------------------------------

\usepackage[top=2.5cm, bottom=2.5cm, left=3.5cm, right=2.5cm]{geometry} %For customizing margins, and specified margins for this document.

\usepackage[turkish]{babel} %Türkçe bölüm isimleri
\usepackage[utf8]{inputenc} %Türkçe karakterler
\usepackage[T1]{fontenc} %Türkçe heceleme

\usepackage{subfig} %Bir girişte birden fazla şekil koymak için
\usepackage{morefloats} %çok sayıda şekil ve tablo kullanılabilmesi için
\usepackage{fixltx2e} %Yazı içerisinde üst karakter ve alt karakter kullanımı için
\usepackage{graphicx} %Şekiller için
\usepackage{xcolor} %Şekiller için
\usepackage{pgf} %Şekiller için
\usepackage{tikz} %Şekiller için (akış şeması vb.)
  \usetikzlibrary{arrows,decorations.pathmorphing,backgrounds,positioning,fit,petri, graphs,mindmap,calendar,backgrounds,shapes.geometric} %Şekiller için
  \tikzstyle{startstop} = [rectangle, rounded corners, minimum width=3cm, minimum height=1cm,text centered, draw=black, fill=red!30] %flowchart start&stop
  \tikzstyle{process} = [rectangle, minimum width=3cm, minimum height=1cm, text centered, text width=4cm, draw=black, fill=orange!30] %flowchart process
  \tikzstyle{process2} = [rectangle, minimum width=3cm, minimum height=1cm, text centered, text width=7cm, draw=black, fill=orange!30] %flowchart process
  \tikzstyle{process3} = [rectangle, minimum width=3cm, minimum height=1cm, text centered, text width=10cm, draw=black, fill=orange!30] %flowchart process
  \tikzstyle{decision} = [diamond, minimum width=1cm, minimum height=1cm, text centered, text width=3cm, aspect=3, draw=black, fill=green!30] %flowchart decision
  \tikzstyle{decision2} = [diamond, minimum width=1cm, minimum height=1cm, text centered, text width=4cm, aspect=3, draw=black, fill=green!30] %flowchart decision
  \tikzstyle{junction}=[circle, inner sep=0pt, minimum size=0mm]
  \tikzstyle{connection} = [circle, minimum width=1cm, minimum height=1cm, text centered,  draw=black, fill=yellow!30]
  \tikzstyle{connection2} = [circle, minimum width=1cm, minimum height=1cm, text centered,  draw=black, fill=blue!20]
\usepackage{longtable} %Birden fazla sayfaya yayılan tablolar için
\usepackage{multirow} %Tablo içinde satırları birleştirmek için
\usepackage{thmtools} %Teorem, tanım, ispat vb. için
  \declaretheorem{definition}
  \declaretheorem{theorem}
  \declaretheorem{proof}
  \declaretheorem{hypothesis}

\usepackage{amssymb} %Matematik için
\usepackage{amsmath} %Matematik için
\usepackage{mathtools} %Matematik için
\usepackage{xfrac} %Matematik için

\usepackage{setspace} %Sayfa boşluklarını düzenlemek için
\usepackage{parskip}
  \setlength{\parskip}{6pt plus 1pt minus 1pt} %Paragraflar arası boşlukları belirlemek için

\usepackage{enumitem} % ``örnek 1.'' gibi etiketli numaralandırma listeleri için

\usepackage{titlesec} %Başlıkları düzenlemek için
  \titleformat{\chapter}[display]
  {\bfseries\large}
  {\filleft\MakeUppercase{\chaptertitlename} \large\thechapter}
  {2ex}
  {\titlerule\vspace{2ex}\filleft}

  \titleformat{name=\chapter,numberless}[display]
  {\bfseries\large}
  {\titlerule}
  {-9ex}
  {\filleft\MakeUppercase}[\vspace{5ex}] \titlespacing*{\chapter}{0pt}{120pt}{6pt}

  \titleformat{\section}{\normalfont\normalsize\bfseries}{\thesection}{6pt}{}
  \titleformat{\subsection}{\normalfont\normalsize\bfseries}{\thesubsection}{6pt}{}
  \titleformat{\subsubsection}{\normalfont\normalsize\bfseries}{\thesubsubsection}{6pt}{}
  \setcounter{secnumdepth}{3}
  \setcounter{tocdepth}{3}

\usepackage{titletoc} %İçindekiler başlığını düzenlemek için

\usepackage{appendix} %Ekler için

  \titlecontents{chapter}
  [0ex]
  {\addvspace{2ex}}
  {\bfseries\@chapapp\ \thecontentslabel\quad \\}
  {\hspace{-0ex}}
  {\hfill\contentspage}
  [\addvspace{0pt}]

  \g@addto@macro\appendices{%
    \addtocontents{toc}{\protect\renewcommand{\protect\@chapapp}{\appendixname}}%
  }


  \renewcommand{\contentsname}{İÇİNDEKİLER} %İçindekiler başlığının düzenlenmesi
  \renewcommand{\bibname}{KAYNAKLAR} %Kaynakça başlığnıın düzenlenmesi

  \newcommand\tocheading{\par\smallskip\MakeUppercase\hfill\textbf{Sayfa}}
  \newcommand\lofheading{\par\smallskip\MakeUppercase\hfill\textbf{Sayfa}\\}
  \newcommand\lotheading{\par\smallskip\MakeUppercase\hfill\textbf{Sayfa}\\}

%--------------------------------------------------------------------------------------------------------------------------------------------------
%---------------------------------------------------Single spacing the lot and lof-------------------------------------------------------
%--------------------------------------------------------------------------------------------------------------------------------------------------
\makeatletter
\def\@chapter[#1]#2{\ifnum \c@secnumdepth >\m@ne
                       \if@mainmatter
                         \refstepcounter{chapter}%
                         \typeout{\@chapapp\space\thechapter.}%
                         \addcontentsline{toc}{chapter}%
                                   {\protect\numberline{\thechapter}#1}%
                       \else
                         \addcontentsline{toc}{chapter}{#1}%
                       \fi
                    \else
                      \addcontentsline{toc}{chapter}{#1}%
                    \fi
                    \chaptermark{#1}%
%                    \addtocontents{lof}{\protect\addvspace{10\p@}}% NEW
%                    \addtocontents{lot}{\protect\addvspace{10\p@}}% NEW
                    \if@twocolumn
                      \@topnewpage[\@makechapterhead{#2}]%
                    \else
                      \@makechapterhead{#2}%
                      \@afterheading
                    \fi}
\makeatother
%--------------------------------------------------------------------------------------------------------------------------------------------------
%-------------------------------------------------------Document Begins Here----------------------------------------------------------
%--------------------------------------------------------------------------------------------------------------------------------------------------

\begin{document}

%--------------------------------------------------------------------------------------------------------------------------------------------------
%---------------------------------------------------------Numberless Pages-------------------------------------------------------------
%--------------------------------------------------------------------------------------------------------------------------------------------------
\shorthandoff{=} %Türkçe babel paketinde = işareti başka amaçla kullanıldığından hatanın önüne geçilmesi için yapılan bir düzenleme. İngilizce hazırlanan belgelerde bu kısmı eklemeye gerek yoktur.
\newgeometry{top=3cm, bottom=0cm}
\shorthandon{=}
\begin{titlepage}
\begin{center}
\begin{singlespacing}
\uppercase{\large T.C.\\ YILDIZ TEKNİK ÜNİVERSİTESİ\\ FEN BİLİMLERİ ENSTİTÜSÜ}\\[4cm]
\uppercase{\large ``TEZİNİZİN ADINI BURAYA YAZIN"}\\[5cm]
\uppercase{\large ``ADINIZI BURAYA YAZIN"}\\[3cm]
\uppercase{\large YUKSEK LİSANS TEZİ\\ ``BÖLÜMÜNÜZÜN ADINI BURAYA YAZIN"\\ ``PROGRAMINIZIN ADINI BURAYA YAZIN"}\\[3cm] %YUKSEK LİSANS TEZI `ni DOKTORA TEZİ ile değiştirebilirsiniz
\uppercase{\large DANISMAN\\ ``DANISMANINIZIN ÜNVANI VE ADINI BURAYA YAZIN"}\\[2cm]
\uppercase{\large ISTANBUL, ``TEZ TESLİM YILINI YAZIN"}
\end{singlespacing}
\end{center}
\end{titlepage}
\restoregeometry
\begin{titlepage}
\begin{center}
\uppercase{\bfseries\large TÜRKİYE CUMHURİYETİ\\ YILDIZ TEKNİK ÜNİVERSİTESİ\\ FEN BİLİMLERİ ENSTİTÜSÜ}\\[1.5cm]
\uppercase{\bfseries\large ``TEZİNİZİN ADINI BURAYA YAZIN"}\\[1.5cm]
\end{center}
\begin{singlespacing}
\textnormal{``BURAYA ADINIZI YAZIN'' tarafından hazırlanan tez çalışması ``BURAYA TEZİNİZİN SUNUM TARİHİNİ YAZIN'' tarihinde aşağıdaki jüri üyeleri tarafından Yıldız Teknik Üniversitesi Fen Bilimleri Enstitüsü ``BURAYA ANABİLİM DALINIZI YAZIN'' Anabilim Dalı'nda \textbf{YÜKSEK LİSANS TEZİ} olarak kabul edilmiştir.}\\[1.5cm] %YÜKSEK LİSANS TEZİ'ni DOKTORA TEZİ `ne çevirebilirsiniz
\textbf{Tez Danışmanı}\\
\textnormal{``DANIŞMANINIZIN ÜNVANI VE ADINI BURAYA YAZIN"\\ Yıldız Teknik Üniversitesi\\''BÖLÜM ADINIZI BURAYA YAZIN"}\\[1.5cm]
\textbf{Jüri Üyeleri}\\
\textnormal{``DANIŞMANINIZIN ÜNVANI VE ADINI BURAYA YAZIN"\\ Yıldız Teknik Üniversitesi\\''BÖLÜM ADINIZI BURAYA YAZIN"}\\[1.5cm]
\textnormal{``1. JÜRİNİN ÜNVAN VE ADINI BURAYA YAZIN"\\ Yıldız Teknik Üniversitesi\\JÜRİ ANABİLİM DALINI BURAYA YAZIN"}\\[1.5cm]
\textnormal{``2. JÜRİNİN ÜNVAN VE ADINI BURAYA YAZIN"\\ ``JÜRİNİN ÜNİVERSİTE ADI''\\"JÜRİ ANABİLİM DALINI BURAYA YAZIN"}\\[1.5cm]
%Jüri satırlarını gerektiği kadar arttırmayı unutmayın!
\end{singlespacing}
\end{titlepage}
---------------------------------------------%ÖNSÖZ SAYFASI%--------------------------------
--------------------------------------------------------------------------------

\begin{titlepage}
\shorthandoff{=}
\topmargin=80pt
\shorthandon{=}
\begin{flushright}
\uppercase{\bfseries{\large önsöz}}\\[0.5cm]
\hrule
\vspace{2ex}%
\end{flushright}
\begin{singlespacing}
"BURAYA ÖNSÖZÜNÜZÜ YAZIN"

\end{singlespacing}
\end{titlepage}

%--------------------------------------------------------------------------------------------------------------------------------------------------
%------------------------------NORMAL SAYILI SAYFALAR--------------------------
%--------------------------------------------------------------------------------------------------------------------------------------------------

\pagenumbering{roman}
\setcounter{page}{4}
\tableofcontents
\addtocontents{toc}{\tocheading}

\chapter*{SİMGE LİSTESİ}
\vspace{-2pt}
\addcontentsline{toc}{chapter}{SİMGE LİSTESİ}
\begin{table}[h]
	\begin{tabular}{p{2cm} l}
    %SATIRLARI İHTİYACINIZA GÖRE DÜZENLEYİN
		\(\textbf{a\textsubscript{ij}}\) & Estimated weight ratio of specimens \(i\) and \(j\) \\
		\(\textbf{E\textsubscript{G}}\) & Set of edges for a graph G\\
		\(\textbf{V\textsubscript{G}}\) & Set of vertices for a graph G\\
	\end{tabular}
\end{table}


\chapter*{KISALTMA LİSTESİ}
\vspace{-2pt}
\addcontentsline{toc}{chapter}{KISALTMA LİSTESİ}
\begin{table}[h]
	\begin{tabular}{p{2cm} l}
    %SATIRLARI İHTİYACINIZA GÖRE DÜZENLEYİN
		\textbf{AHP} & Analytic Hierarchy Process\\
		\textbf{CI} & Consistency Index\\
	\end{tabular}
\end{table}

{%
\let\oldnumberline\numberline%
\renewcommand{\numberline}{\figurename~\oldnumberline}%
\listoffigures%
\addtocontents{lof}{\lofheading}
}

{%
\let\oldnumberline\numberline%
\renewcommand{\numberline}{\tablename~\oldnumberline}%
\listoftables%
\addtocontents{lot}{\lotheading}
}


\chapter*{\"{O}ZET}
\vspace{-30pt}
\begin{center}
\uppercase{\large{\bfseries{"TEZİNİZİN ADI"}}}\\[1cm]

"ADINIZ"\\[1cm]

"BÖLÜMÜNÜZÜN ADI"\\
Yüksek Lisans Tezi\\[1cm] %"DOKTORA TEZİ" OLARAK DEĞİŞTİREBİLİRSİNİZ

Danışman: "DANIŞMAN ÜNVAN VE ADI"\\[1cm]
\end{center}

\begin{singlespacing}
"BURAYA ÖZETİNİZİ YAZIN"

\vspace{2.5cm}
\textbf{Anahtar Kelimeler:} "VİRGÜL İLE AYIRARAK 5 ADET ANAHTAR KELİME YAZINIZ"
\end{singlespacing}

\vspace{1ex}
\hrule
\begin{flushright}
\textbf{YILDIZ TEKNIK UNIVERSITESI \\ FEN BILIMLERI ENSTITUSU}
\end{flushright}

\chapter*{ABSTRACT}
\vspace{-30pt}
\begin{center}
\uppercase{\large{\bfseries{"YOUR THESIS NAME HERE"}}}\\[1cm]

"YOUR NAME HERE"\\[1cm]

"YOUR DEPARTMENT NAME HERE"\\
MSc. Thesis\\[1cm] %CHANGE WITH "PhD. THESIS" IF NECESSARY

Advisor: "WRITE YOUR ADVISOR'S PREFIX AND NAME HERE"\\[1cm]
\end{center}

\begin{singlespacing}
"WRITE YOUR ABSTRACT IN ENGLISH HERE"

\vspace{3cm}
\textbf{Keywords:} "WRITE YOUR KEYWORDS HERE WITH USING COMMA AS SEPARATOR"
\end{singlespacing}

\vspace{1ex}
\hrule
\begin{flushright}
\textbf{YILDIZ TECHNICAL UNIVERSITY \\ GRADUATE SCHOOL OF NATURAL AND APPLIED SCIENCES}
\end{flushright}


%--------------------------------------------------------------------------------------------------------------------------------------------------
%-------------------------------------------ANA METİN------------------------------------------------------------------
%--------------------------------------------------------------------------------------------------------------------------------------------------

% \newpage
\pagenumbering{arabic}

\chapter{Giriş} %BU BÖLÜMÜN ADINI DEĞİŞTİRMEYİN. FBE TARAFINDAN İSTENİLEN ZORUNLU BAŞLIKTIR
"GİRİŞ KISMINI BURAYA YAZIN"

\section{Literatür Özeti} %BU BÖLÜMÜN ADINI DEĞİŞTİRMEYİN. FBE TARAFINDAN İSTENİLEN ZORUNLU BAŞLIKTIR
\label{sec: literatur ozeti}
"LİTERATÜR ÖZETİNİZİ BURAYA YAZIN"

\section{Tezin Amacı} %BU BÖLÜMÜN ADINI DEĞİŞTİRMEYİN. FBE TARAFINDAN İSTENİLEN ZORUNLU BAŞLIKTIR
\label{sec: amaç}
"TEZİN AMACINI BURAYA YAZINIZ"

\section{Orijinal Katkı} %FBE TARAFINDAN İSTENİLEN ZORUNLU BAŞLIKTIR. BU BAŞLIĞI AŞAĞIDAKİ 2 SENEÇEKTEN SİZE UYGUN OLANLA DEĞİŞTİREBİLİRSİNİZ
		%   1. "Orijinal Katkı"
                %     2. "Hipotez"
\label{sec: original}
"BURADA SEÇTİĞİNİZ BAŞLIĞIN ALTINI DOLDURUNUZ"

\chapter{``BURAYA BÖLÜM ADI YAZILACAK''} %TEZİNİZİN BÖLÜM ADI
"BURAYA YAZABİLİRSİNİZ"
\section{"BÖLÜM 2 BAŞLIK 1"}
\label{sec: "BAŞLIK 1"}

"AŞAĞIDAKİ TABLO, TABLO YAZISININ GÖSTERİMİNİ ÖRNEKLENDİRMEK AMACIYLA KONULMUŞTUR. BÜTÜN TABLO YAZILARI TABLONUN ÜSTÜNDE OLMALIDIR. TABLO YAZISI ETİKETİNİ İSTEDİĞİNİZ ŞEKİLDE DÜZENLEYEBİLİRSİNİZ; ANCAK YERİNİ DEĞİŞTİRMEMEYE DİKKAT EDİN!''
\begin{table}
	\begin{center}
	\caption{Örnek Tablo\label{tab: etiket1}}
		\begin{tabular}{| c | c |}
		\hline
		HEADER 1  & HEADER 2 \\
		\hline
		1 & 2 \\
		\hline
		3 & 4 \\
		\hline
		5  & 6 \\
		\hline
		7 & 8 \\
		\hline
		\end{tabular}
	\end{center}
\end{table}

"AŞAĞIDAKİ ŞEKİL ÖRNEK TİKZ ŞEKİLLERİ ORTAMIDIR. DAHA FAZLA BİLGİ İÇİN TIKZ\&PGF, SMARTDİAGRAM MANUALLERİNİ ARATINIZ. ŞEKİLLERİN YAZISI HER ZAMAN ŞEKİLLERİN ALTINDA OLMALIDIR.''

\begin{figure}[h]
	\begin{center}
		\begin{tikzpicture}
			"TIKZ \& PGF KODU BURAYA. ŞEKİL YAZISININ YERİNİ DEĞİŞTİRMEYİN."
		\end{tikzpicture}
	\end{center}
	\caption{"Şekil yazısı" \label{fig: "şekil etiketi"}}
\end{figure}

\subsection{"ALTBAŞLIK 2.1.1"}
\label{sec: "BURAYA ETİKETK"}
"UNUTMAYIN! FBE SADECE 4 ALT BAŞLIĞA KADAR KULLANMA İZİNİ VERİYOR YANİ KULLANABİLECEĞİNİZ EN ALT BAŞLIK TÜRLERİ AŞAĞIDAKİ GİBİDİR. İHTİYACINIZA GÖRE KULLANABİLİRSİNİZ.:"
\begin{verbatim}
\subsubsection{}
\subsubsubsection{}
\end{verbatim}

"AŞAĞIDA ÖRNEK BİR ŞEKİL ALANI BULUNMAKTADIR. ŞEKİL YAZISININ YERİNİ DEĞİŞTİRMEYİN. ŞEKİL YAZISI ŞEKLİN ALTINDA OLMALIDIR. AYRICA KULLANACAĞINIZ ŞEKİLİN TEX DOSYASIYLA AYNI ADRESTE OLMASINA DİKKAT EDİN. AYRINTILI BİLGİ İÇİN WİKİBOOKS'TA LATEX KİTABINI İNCELEYEBİLİRSİNİZ.''

\begin{figure}[!h]
\begin{center}
\includegraphics[width=0.7\textwidth]{"ŞEKİL ADI.ŞEKİL UZANTISI"}
\end{center}
\caption{"ŞEKİL YAZISI" \label{fig: "ŞEKİL ETİKETİ"}}
\end{figure}


%--------------------------------------------------------------------------------------------------------------------------------------------------
%---------------------------------------------------------------Bibliography------------------------------------------------------------------
%!!!!!!!!!!!!!!!!!!BURAYI DEĞİŞTİRMEYİN!!! SADECE MASTERS.BIB DOSYASINI DÜZENLEYİN. BİBTEX KULLANIMI İÇİN WİKİBOOKS LATEX KİTABINI İNCELEYİN.!!!!!!!!!!!!!!!!

\bibliography{Masters}{}
\bibliographystyle{yildiz_fbe}
\addcontentsline{toc}{chapter}{KAYNAKLAR}
%--------------------------------------------------------------------------------------------------------------------------------------------------
%-------------------------------------EKLER-------------------------------------------------------------------
%--------------------------------------------------------------------------------------------------------------------------------------------------

\begin{appendices}
\chapter{"EK 1 ADI BURAYA"}
\label{appx: "ETİKET BURAYA"}
\onehalfspacing

"EĞER TEZİNİZDE EK KULLANMAYACAKSANIZ BU KISMI \% İŞARETİYLE ETKİSİZ HALE GETİRİN (AÇIKLAMA HALİNE GETİRİN). BİRDEN FAZLA EK BÖLÜMÜ İÇİN İSTEDİĞİNİZ SAYIDA ARTTIRABİLİRSİNİZ."

\chapter{"EK 2 ADI BURAYA"}
\label{sec: "ETİKET BURAYA"}


\end{appendices}

%--------------------------------------------------------------------------------------------------------------------------------------------------
%-----------------------------ÖZGEÇMİŞ---------------------------
%--------------------------------------------------------------------------------------------------------------------------------------------------
\chapter*{ÖZGEÇMİŞ}
\addcontentsline{toc}{chapter}{ÖZGEÇMİŞ}

\begin{table}[!h]
\begin{tabular}{l l}
\multicolumn{2}{l}{\bfseries{KİŞİSEL BİLGİLER}}\\[2ex]
\bfseries{Adı Soyadı} & "BURAYA ADINIZ"\\[2ex]
\bfseries{Doğum Tarihi ve Yeri} & "DOĞUM GÜNÜ" - "DOĞUM YERİ"\\[2ex]
\bfseries{Yabancı Dili} & "VİRGÜL İLE AYIRARAK BİLDİĞİNİ YABANCI DİLLERİ YAZINIZ"\\[5ex]
\end{tabular}

\begin{tabular}{l l l p{2cm}}
\multicolumn{4}{l}{\bfseries{ÖĞRENİM DURUMU}}\\[2ex]
\bfseries{Derece} & \bfseries{Alan} & \bfseries{Okul/üniversite} & \bfseries{Mezuniyet Yılı} \\[2ex]
PhD. & Dept. of "DEPARTMENT NAME HERE" & "UNIVERSITY HERE" & "DATE OF GRADUATION" \\ [2ex]
Master's & Dept. of "DEPARTMENT NAME HERE" & "UNIVERSITY HERE" & "DATE OF GRADUATION" \\[2ex]
Undergraduate & Dept. of "DEPARTMENT NAME HERE" & "UNIVERSITY HERE" & "DATE OF GRADUATION" \\[2ex]
Highschool & "HIGHSCHOOL HERE" & "DATE OF GRADUATION" \\[5ex]
\end{tabular}

\begin{tabular}{l p{9.5cm} l}
\multicolumn{3}{l}{\bfseries{İŞ TECRÜBESİ}}\\[2ex]
\bfseries{Yıl} & \bfseries{Firma/Kurum} & \bfseries{Görevi} \\[2ex]
"START" - \dots & "CORPORATION OR INSTUTE HERE" & "YOUR JOB TITLE HERE" \\[2ex] %THIS ROW IS FOR YOUR ONGOING WORK EXPERIENCE. IF YOU ARE NOT EMPLOYED IN THE PRESENT REPLACE "\dots" WITH AN END DATE
"START" - "END" & "CORPORATION OR INSTUTE HERE" & "YOUR JOB TITLE HERE" \\[2ex]
"START" - "END" & "CORPORATION OR INSTUTE HERE" & "YOUR JOB TITLE HERE" \\[2ex]
"START" - "END" & "CORPORATION OR INSTUTE HERE" & "YOUR JOB TITLE HERE" \\[2ex]
%IF YOUR WORK EXPERIENCE EXCEEDS ONE PAGE CONTINUE TO NEXT TABLE ENVIRONMENT
\end{tabular}
\end{table}

\begin{table}[!ht]
\begin{tabular}{l p{9.5cm} l}
\multicolumn{3}{l}{\bfseries{İŞ TECRÜBESİ}}\\[2ex]
"START" - "END" & "CORPORATION OR INSTUTE HERE" & "YOUR JOB TITLE HERE" \\[2ex]
"START" - "END" & "CORPORATION OR INSTUTE HERE" & "YOUR JOB TITLE HERE" \\[2ex]
"START" - "END" & "CORPORATION OR INSTUTE HERE" & "YOUR JOB TITLE HERE" \\[5ex]
\end{tabular}
\begin{tabular}{p{15.5cm}}

\bfseries{YAYINLAR} \\[2ex]
\bfseries{Bildiri} \\[2ex]
\textbf{1.} "PUBLISHED WORK 1"\\[2ex]
\textbf{2.} "PUBLISHED WORK 2"\\[2ex]
\textbf{3.} "SAYIYI GÖSTERİLEN ŞEKİLDE ARTTIRABİLİRSİNİZ. GÖSTERİMİ KAYNAKÇA GÖSTERİMİNE UYGUN OLARAK YAPINIZ. AYRICA BU KISIM İÇİN AYRINTILI BİLGİYİ FBE'NİN ŞABLONUNDAN İNCELEYİNİZ."\\[5ex]
\end{tabular}
\end{table}
%--------------------------------------------------------------------------------------------------------------------------------------------------
%--------------------BELGE SONU----------------------------------------
%--------------------------------------------------------------------------------------------------------------------------------------------------
\end{document}
